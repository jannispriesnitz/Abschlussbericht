\section{Einleitung}
in meiner Praxispahase...
\subsection{Motivation}
meine Praxisphase habe ich gemacht bei conti weil..

\section{Das Unternehmen}

Die Continental Automitive AG entwickelt und baut am Standort Babenhausen Kombiinstrumente, Head-up Displays (HUDs) und ähnliche informationstechnische Komponenten im Bereich von automotive embedded Systems für zahlreiche Automobilhersteller weltweit. Dabei wird nahezu jedes Bauteil und jede Softwarekomponente  selbst entwickelt und produziert. 

\subsection{Mein Tätigkeitsbereich}
Meine Praxisphase absolviere ich in der \textbf{Devision "`Interior"'} im \textbf{Bereich "`Instrumentation \& Driver HMI"'}, welcher sich mit der Entwicklung von modernen Kombiinstrumenten, Head-up Displays und Steuerungseinheiten im Cockpit eines Autos beschäftigt.
\paragraph{Die Einheit "`Core Development Software (CDS)"'} entwickelt eine Basis für die Anwendungsentwicklung durch einheitliche Technologien, Standards und Lösungen, die das Durchführen von Softwareprojekt unter einheitlichen Standards und Prozessen und Plattformen unterstützt. Außerdem werden neue Technologien erforscht und bestehende verbessert.  
\paragraph{Die Gruppe "`Flash/Bootloader"'} schließlich entwickelt Flashloader Systeme, die für das aufspielen und das starten des Systems verantwortlich sind, sowie Packaging- und Diagnosesysteme in Verbindung mit Flashspeichern. Ziel ist es einen einheitlichen, einfachen und sicheren Flashprozess sowohl in der Fertigung, als auch beim Endkunden zu gewährleisten und Fehlfunktionen schnell zu diagnostizieren, sowie eine lange Lebensdauer der Flashspeicherbausteine zu erreichen.



\subsection{Produkte}
\subsection{Unternehmensbereiche}
%weitere unterpunkte für CDS Automotive etc. 
\subsection{organisatorische Einbettung} % klingt scheiße
% + erklären, warum ich im CDS sitze jedoch nichts mit denen direkt zu tun habe
\section{Das Projekt}
CANKrypto blabla
\subsection{Allgemeine Beschreibung}
\subsection{Meine Aufgabe} %ggf. redundant
%\subsection{Hintergrund}
%muss hier noch nicht unnbeding rein. 
Meine Aufgabe ist es ein Konzept für eine \textbf{sichere Kommunikation über das Controller Area Network (CAN)} zu erstellen. Dazu soll ein bestehendes CAN-Demoprogramm umfassend umgebaut werden, kryptografische Verfahren und Parameter ausgewählt werden und diese in das CAN-Setup integriert werden. Außerdem soll die Software auf verschiedene Hardwar- und Betriebssystemen integriert und umfassende Performancetests durchgeführt werden. 

\subsection{Ziele}
Ziel des Konzeptes ist es die Machbarkeit der sicheren Kommunikation über das CAN Protokoll zu beurteilen. Dazu sollen kryptografische Mittel nach aktuellem Stand der Technik ausgewählt und ein Vorschlag für ein Kommunikationsprotokoll gemacht werden, das eine für die Anwendungsschicht transparente Kommunikation gewährleistet. Die gewählten Ansätze sollen in Software umgesetzt werden und hinsichtlich Performance, Speicherverbrauch und Portabilität geprüft werden. 


\subsection{Eingesetzte Technologien}
C 
Visuual Studio 
CrypoPP, CryptLib, OpenSSL


\subsection{Anforderungen}
%M-N Kommunikation
%authentisch
\subsection{Konzepte und Lösungsansätze}
wurden komplett von mir entwickelt

\subsection{Aktueller Stand} % was wurde realisiert?
Prototyp 
Messungen
\subsection{Bewertung des Ergebnisses} 
%ggf. redundant zu unten

- Viel geschafft 
viel muss noch getan werden
\subsection{Ausblick}
Auch wenn dies ein Abschlussbericht ist, möchte ich kurz auf meine weiteren Tätigkeiten im Rahmen der noch verbleibenden Zeit in der Praxisphase geben. Außerdem möchte ich noch kurz einen generellen Ausblick zu der von mir bearbeiteten Thematik geben. 

\section{Evaluation der Praxisphase}
\subsection{Das Unternehmen}
\subsection{Lessons Learned}
Crypto API schneller wählen+
direkt c entwickeln
\subsection{Persöhnliche Einschätzung des Lernerfolgs}

%Abkürzungsverzeichnis brauche ich wohl nicht 

% Glossar 

%Quellennachweis / Literaturverzeichnis
% kann aus dem Expose genommen und leicht angepasst werden 
blub