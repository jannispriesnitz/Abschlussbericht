% GITHub Adresse: 
% https://github.com/jannispriesnitz/Abschlussbericht.git


\documentclass[a4paper,12pt]{article}

%----------------- PDF CONFIG ----------------- %
\pdfinfo{    
     /Title (Exposè) 
     /Subject   (Exposè)    
     /Author  (Jannis Priesnitz) 
     /Keywords   (Stichwort1,Stichwort2)      
} 




%----------------- PAKETE INKLUDIEREN ----------------- %

\usepackage{geometry} % Packet für Seitenrandabständex und Einstellung für Seitenränder
\usepackage[ngerman]{babel} % deutsche Silbentrennung

\usepackage{booktabs} %entzerrt die Tabellenzeilen und bietet verschieden dicke Unterteilungslinien
\usepackage{longtable} % Tabellen können sich nicht über mehrere Seiten 
\usepackage{graphicx} % kann LaTeX Grafiken einbinden
\usepackage[utf8]{inputenc}  % Umlaute unter Linux

\usepackage[T1]{fontenc} % Zeichenencoding
\usepackage{lmodern} % typographische Qualität 
\frenchspacing % Schaltet den zusätzlichen Zwischenraum ab
\usepackage{fix-cm}
\usepackage{hyperref} % verwandelt alle Kapitelüberschriften, Verweise aufs Literaturverzeichnis und andere Querverweise in PDF-Hyperlinks
\usepackage{color}
\usepackage{url}
\usepackage{float} %force position
\usepackage{caption, booktabs}
\usepackage[nottoc]{tocbibind}


% Service for todonotes
%\usepackage{lipsum}                     % Dummytext
\usepackage{xargs}                      % Use more than one optional parameter in a new commands
\usepackage[pdftex,dvipsnames]{xcolor}  % Coloured text etc.
%Package todonotes

%NOTE: um alle Notizen unsichtbar zu machen als Parameter disable einfügen
\usepackage[colorinlistoftodos,prependcaption,textsize=tiny]{todonotes}
\newcommandx{\unsure}[2][1=]{\todo[linecolor=red,backgroundcolor=red!25,bordercolor=red,#1]{#2}}
\newcommandx{\change}[2][1=]{\todo[linecolor=blue,backgroundcolor=blue!25,bordercolor=blue,#1]{#2}}
\newcommandx{\info}[2][1=]{\todo[linecolor=OliveGreen,backgroundcolor=OliveGreen!25,bordercolor=OliveGreen,#1]{#2}}
\newcommandx{\improvement}[2][1=]{\todo[linecolor=Plum,backgroundcolor=Plum!25,bordercolor=Plum,#1]{#2}}
\newcommandx{\thiswillnotshow}[2][1=]{\todo[disable,#1]{#2}}

\todo[disable]

% für Listings
\usepackage{listings}
\lstset{numbers=left, numberstyle=\tiny, numbersep=5pt, stepnumber=4, keywordstyle=\color{black}\bfseries\itshape, stringstyle=\ttfamily,showstringspaces=false,basicstyle=\footnotesize,captionpos=b}
% Trick zum leichten einbingen des Logos in den Titel 
\lstset{language=java}
	\title{\includegraphics[width = 0.6\textwidth]{Abb/logo_fbi.jpg} \\[1cm] \huge 
	% HIER TITEl
		Bericht zu meiner Praxisphase
		}
	\author{\\ vorgelegt von: \\[5mm]Jannis Priesnitz, 729341}

	\date{}

%stellt eigene anpassungen f�r \maketitle zur Verf�gung
\usepackage{titling}
%eigener untertitel kann auch in die aux Datei
\newcommand{\subtitle}[1]{%
	\posttitle{%
		\par\end{center}
	\begin{center}\large#1\end{center}
	\vskip0.5em}%
}
	\subtitle{im Rahmen des Informatikstudiums an der Hochschule Darmstadt}








%----------------- LAYOUT SETZEN ----------------- %
%\geometry{left=2cm, right=2cm, top=2.5cm, bottom=2cm}


% eigene Anpassungen im Skript

\setlength{\skip\footins}{10mm} %setzt den Abstand zwischen Text und Fußnote

\begin{document}
	



\maketitle		
%muss unter maketitle stehen, da latex denkt maketitle ist eine eigene Seite. 

\thispagestyle{empty}

\begin{center}
	
	Referent: Prof. Moore \\[0.5cm]
	Coreferent: Prof. Wiedling \\[2cm]
	
	im Sommersemester 2015 \\[5mm]
	Griesheim, den \today
	
	
\end{center}


\vspace{1.4cm}

%----------------- FARBEN DEFINIEREN ----------------- %
\definecolor{gray}{gray}{0.95} % Listingsbackground

\linespread {1.25}\selectfont %1.25 da er von Haus aus 1.2 ist und 1,25 * 1,2 = 1,5 isch



%\begin{center}
	
%	\rule[1cm]{0.88\linewidth}{0.1pt} %länge der Linie = Abstractbreite
%\end{center}

%\begin{abstract}
%\end{abstract}


\section*{Eigenständigkeitserklärung}
Ich versichere hiermit, dass ich den vorliegende Praxisbericht selbständig verfasst
und keine anderen als die im Literaturverzeichnis angegebenen Quellen benutzt habe.
Alle Stellen, die wörtlich oder sinngemäß aus veröffentlichten oder noch nicht
veröffentlichten Quellen entnommen sind, sind als solche kenntlich gemacht. Die
Zeichnungen oder Abbildungen in dieser Arbeit sind von mir selbst erstellt worden
oder mit einem entsprechenden Quellennachweis versehen. Diese Arbeit ist in
gleicher oder ähnlicher Form noch bei keiner anderen Prüfungsbehörde eingereicht
worden.
\\[5mm]
Griesheim, den \today

<Name und Unterschrift>

\newpage		
\tableofcontents

\section{Problemstellung}
Das Controller Area Network (CAN) wird in einer Vielzahl von Systemen, vor allem im Bereich eingebetteter Systeme, eingesetzt. Eine Vielzahl davon enthalten sicherheitskritische Komponenten, deren Ausfall weitreichende Konsequenzen nach sich ziehen sowie Risiken für den Anwender darstellen. Beispiele hierfür sind der Ausfall des Antiblockiersystems während der Autofahrt oder das Einschleusen von Schadsoftware in Kraftwerken oder Fabriken.
 Nach aktuellem Stand der Technik wird das Controller Area Network komplett ohne eine Verschlüsselung der übertragenen Informationen und Authentifikation der Teilnehmer verwendet. Die Sicherheit beruht derzeit ausschließlich auf der Abgeschlossenheit des Systems sowie des meist geheimgehaltenen Softwareprotokolls. 

Vor allem moderne Autos bieten jedoch die Möglichkeit durch "`Basteleien"' Busteilnehmer auszutauschen und so Schadsoftware in das System gelangen zu lassen. 

Des weiteren können Komponenten, die andere Schnittstellen wie z.B. W-Lan besitzen, übernommen werden und so über das CAN\footnote{Da es sich bei CAN um die Abkürzung für Controller Area Network handelt, wird CAN im Folgenden so verwendet.} sicherheitskritische Komponenten infizieren. 

\section{Erkenntnisinteresse}
Um die Sicherheit von technischen Anlagen und Automobilen, die CAN nutzen, zu gewährleisten, muss darüber nachgedacht werden, wie dies auf einem akzeptablen Niveau geleistet werden kann. Die Idee einer Verschlüsselung wurde bereits durch verschiedene Firmen betrachtet und soll hier aufgenommen und um den Aspekt der Instanzauthentifikation ergänzt werden. Dies ist speziell für Systeme, in denen Komponenten einfach von unautorisierten Menschen ausgetauscht werden können, wie z.B. im Auto, von großer Bedeutung.

Das zu erstellende Konzept zeigt den Aufwand für solch ein Vorhaben auf und gibt eine Einschätzung über deren Machbarkeit. Hierzu werden sowohl gängige kryptografische Mittel von mir ausgewählt, um maximale Sicherheit zu gewährleisten, als auch Messungen an einer exemplarisch ausgewählten Untermenge mit geringerem Sicherheitsniveau (einem Subset an Sicherheit) von mir durchgeführt. 

Meine Aufgabe besteht zusammenfassend darin, die bereits gewonnen Erkenntnisse in dem Bereich der Kryptografie und der Embedded Systems zusammen zu bringen und einen fundiertes Konzept über die Möglichkeiten zu geben. 

\section{Fragestellung}
Zentrale Fragestellung der Arbeit ist es, ob es effizient möglich ist, \emph{eine} verschlüsselte und authentische Kommunikation in dem CAN durchzuführen. %Fußnote einfügen 
\newline
Dies unterteilt sich zunächst prinzipiell in den Aufwand für die Initialisierung der Kommunikation und den zusätzlichen Aufwand für verschlüsselte Kommunikation zur Laufzeit.
\newline
Weiter soll die Frage geklärt werden, \emph{welche Mittel} eingesetzt werden müssen, um ein maximales Sicherheitsniveau nach dem Stand der Technik einzusetzen. Hieraus resultiert direkt die Aufgabe, geeignete kryptografische Mittel für die gegebenen Hardwarevoraussetzungen auszuwählen und eine Abschätzung der Performance zu geben.
\newline
Die theoretischen Betrachtungen sollen anhand der Frage, \emph{wie performant} ein von mir gewähltes %kann ich das so schreiben?
und prototypisch implementiertes Subset an Sicherheit sein kann, untermauert werden.
\newline
Die Eigenschaften und Vorzüge des Netzwerkes müssen hierbei soweit, wie möglich gewahrt bleiben und Einschränkungen hinsichtlich Kommunikationsteilnehmern sollen genau beschrieben und begründet werden.
Ferner wird kurz auf die Hardwarevoraussetzungen eingegangen, die ein Kommunikationsteilnehmer mitbringen sollte, um die gewählten Algorithmen durchführen zu können. 

\section{Zielsetzung}
Es soll gezeigt werden, dass eine verschlüsselte und authentische Kommunikation nach dem Stand der Technik über das CAN grundsätzlich möglich ist. Darüber hinaus soll auf die besonderen Randbedingungen bei einem Einsatz im automotive embedded Umfeld eingegangen werden. Außerdem sollen gängige kryptografische Mittel ausgewählt werden, um dies zu realisieren und deren Wahl kurz begründet werden. 
An einer prototypischen Implementierung einer Teilmenge sollen Messwerte genommen werden, anhand derer eine differenzierte Aussage über die Realisierbarkeit einer sicheren Kommunikation getroffen werden kann. Aus diesen Ergebnissen wird anhand von Komplexitätsabschätzungen auf die Realisierbarkeit und Performance der maximalen Sicherheit geschlossen. 

\section{Theoriebezug und Modellbildung}
Die Arbeit basiert auf der Theorie, dass jede Kommunikation durch geeignete Mittel verschlüsselt und authentisch statt finden kann, welche letztendlich auf der mathematischen Gruppentheorie basiert. 
Es wird ein Modell gebildet, welches die Kommunikation zwischen mehreren Kommunikationspartnern betrachtet und in ähnlicher Weise (in Bezug auf Chipperformance) in aktuellen Systemen, die das CAN-Protokoll nutzen, zum Einsatz kommen könnte. Das Modell ist hinsichtlich Kommunikationsteilnehmern und kryptografischer Sicherheit gegenüber der Realität eingeschränkt.
Es soll ein Multiproducer - Multiconsumer Modell betrachtet werden, in dem die Kommunikation von einem zentralen Server geregelt wird. 
Dieser Server verwaltet eine oder mehrere Domains, deren zwei oder mehr Clients angehören. 


%\section{Forschungsstand}
%Sowohl das Feld der Kryptografie, als auch der Entwicklung von Software speziell für eingebettete Systeme waren Gegenstand vieler Forschungen. 
%Arbeiten beschäftigen sich mit der sicheren Kommunikation über Geräte mit sehr wenig Leistung und über Schnittstellen, die über eine geringe Nutzlast und lange Nachrichtenlaufzeiten verfügen.



\section{Methode}
Es wird ein Prototyp erstellt, der ein Subset einer sicheren Kommunikation darstellt. Anhand dieser Implementierung werden verschiedene Algorithmen mit unterschiedlichen Randbedingungen getesteten. Aus den Ergebnissen werden Empfehlungen für alternative Algorithmen und Verfahren und eine Abschätzung der Effizienz dieser gegeben. 

%\section{Materialübersicht}
%TBC
\section{Vorläufige Gliederung}
Nach aktuellem Erkenntnissen gestaltet sich eine Gliederung folgendermaßen:
\begin{enumerate}
	\item Einleitung / Hintergrund

	\item  Aufgabenstellung
	\subitem   Authentische Kommunikation
	\subitem  Verschlüsselte Kommunikation
	
	\item  Randbedingungen der Automotive Embedded Welt und von CAN
	\subitem  Rechenleistung 
	\subitem  Speicherverwaltung
	\subitem  Betrachtungen zu Multithreadingeigenschaften
	\subitem  Ausfallsicherheit

	\item  Konkretes Design der prototypischen Implementierung
	\subitem  Authentisierung
	\subitem  Schlüsseltausch
	\subitem  Verschlüsselung 

	\item Performancemessungen
	\subitem  Setup Zeit
	\subitem  Lauftzeit einer Nachricht
	\subitem  Speicher
	
	\item Kryptografisches System mit maximaler Sicherheit
	\subitem Vorstellung des Systems
	\subitem Abschätzung der Realisierbarkeit aufgrund der Messungen am Prototyp
	\item  Fazit und Ausblick

	\end{enumerate}


\section{Literaturverzeichnis}
Die bis jetzt von mir recherchierten,  zentralen Quellen sind im Folgenden angegeben. Weitere Quellen sind noch zu beschaffen. 
\nocite{*}
\bibliographystyle{plain}
\bibliography{Literatur/Literaturverzeichnis}

\section{Zeitplan}


\begin{tabular}{|p{5cm}|p{9cm}|}
	\hline 24.04.2015 & Anmeldugn zu Bachelorarbeit \\ 
	\hline \today & Abgabe des Exposès \\ 
	\hline \today - 15.05.2015 & Fertigstellung des Prototypen (und ende Praxisphase) \\ 
	\hline 15.05.2015 - 07.06.2015 & Messungen am Prototyp und Auswertung \\ 
	\hline 07.06.2015 - 14.06.2015 & Design eines Systems mit maximaler Soicherheit \\ 
	\hline 15.06.2015 - 29.06.2015 & Betrachtungen zu den Verausetzungen für einzusetztende Systeme \\ 
	\hline 29.06.2015 - 17.07.2015 & Schlussfolgerung der Performance des Prototype auf das System maximalere SIcherheit \\ 
	\hline 18.07.2015 - 23.07.2015 & Korrektur und Verbesserungen \\ 
	\hline 24.07.2015 & Abgabe der Bachelorarbeit \\ 

	\hline 
\end{tabular} 


\nocite{*}
\bibliographystyle{plain}
\bibliography{Literatur/Literaturverzeichnis}

\listoftodos
	
\end{document}